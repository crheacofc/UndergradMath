\documentclass[10pt,a4paper]{article}
\usepackage[utf8]{inputenc}
\usepackage{amsmath}
\usepackage{amsfonts}
\usepackage{amssymb}
\author{Carter Rhea}
\title{Axiomatic Geometry Homework 1: Propositions 2.3,2.4,2.5}
\begin{document}
\maketitle
\begin{enumerate}



\item Proposition 2.4 For every point there is at least one line not passing through it
\begin{enumerate}
\item Let $P_0$ be an arbitrary point.
\item Then there exists two other points, $P_1$ and $P_2$, such that $P_0$,$P_1$, and $P_2$ are not colinear. (I-3)
\item There exists a line, $\overleftrightarrow{P_1 P_2}$. (I-1)
\item Thus, $\overleftrightarrow{P_1 P_2}$ does not pass through $P_0$. (Steps 2 and 3).
\item Therefore, since $P_0$ was an arbitrary point, for every point there is at least one line not passing through it.
\end{enumerate}
\item For every point, P, there exists at least two lines through it.
\begin{enumerate}
\item Let $P_0$ be an arbitrary point.
\item Then there exists two other points, $P_1$ and $P_2$, such that $P_0$,$P_1$, and $P_2$ are not colinear. (I-3)
\item There exist lines $\overleftrightarrow{P_0 P_1}$ and $\overleftrightarrow{P_0 P_2}$ (I-1)
\item Thus, since $P_0$ was arbritrary, for every point, P, there exists at least two lines through it.
\end{enumerate}
\item 9d) Fix a sphere in Euclidean three-space. Two points on the sphere
are called antipodal if they lie on a diameter of the sphere; e.g., the
north and south poles are antipodal. Interpret a "point" to be a set
{P, PI} consisting of two antipodal points on the sphere. Interpret
a "line" to be a great circle C on the sphere. Interpret a "point"
{P, PI} to "lie on" a "line" C if one of the points P, pI lies on the
great circle C (then the other point also lies on C).
 Determine if I-1,I-2,and I-3 hold. Also determine the parallel property.
\begin{enumerate}
\item I-1: I-1does hold because it is possible to find two antipodal pairs which are  incident on the same great circle. If you find two antipodal pairs and draw a great circle through only one point in each pair, you will end up drawing the line through the other point in the pair, thus connecting the two pairs with a line.
\item I-2: I-2 would hold because great circles would go through at least two antipodal points.
\item I-3: I-3 hold because it is possible to find 3 points which are not colinear since you could pick 3 points lying on different diameters and then not be on the same great circle.
\item Parallel: The space would have Elliptical parallel property. If you had two antipodal pairs, two points, and draw a great circle through them, which would be a diameter, then they would have to cross and thus, there would be no parallels. 
\end{enumerate}
\end{enumerate}

\end{document}